\documentclass[11pt,DIV=13,a4paper,headinclude,german]{scrartcl}
\usepackage[ngerman]{babel}
\usepackage[utf8]{inputenc}
\usepackage[OT1]{fontenc}
\usepackage{lmodern}
\usepackage{upgreek}
%\usepackage{graphicx}
%\usepackage[figuresright]{rotating}     % l344dt auch graphicx
%\usepackage{xcolor}
\usepackage{amsmath}
\usepackage{amssymb}
\usepackage{setspace}
\usepackage{scrpage2}
\usepackage{textcomp}
\usepackage{ragged2e}
%\usepackage{booktabs}
%\usepackage{threeparttable}
%\usepackage{rotating}
%\usepackage{enumitem}
%\usepackage{color}
%\usepackage{etoolbox}
%\usepackage{bibentry}
%\usepackage{placeins}
%\usepackage{cite}
\usepackage[german]{babelbib}

%  \clubpenalty = 10000
%  \widowpenalty = 10000

%\linespread{1.25}
% \KOMAoptions{DIV=last}

%%%%%%%%%%%%%%%%%%%%%%%%%%%%%%%%%%%%%%%%%%%%%%%%%%%%%%%%%%%%%%%%%%%%%%%%%%%%%%%%%%%%%%%%%%%%%%%%%%%%%%%%%%%%%%%%%%%%%%%%%%%%%%%%%%%%%%%%%%%%%%%%%%%%%%%%%%%%%%%%%%%%%%%%%%%%%%%%%%%%%%%%%%%%%%%%%%%%%%%%%%%%%%%%%%%%%%%%%%%%%%%%%%%%%%%%%%%%%%%%%%%%%%%%%%%%%%%%%%%%%%%
%%%%%%%%%%%%%%%%%%%%%%%%%%%%%%%%%%%%%%%%%%%%%%%%%%%%%%%%%%%%%%%%%%%%%%%%%%%%%%%%%%%%%%%%%%%%%%%%%%%%%%%%%%%%%%%%%%%%%%%%%%%%%%%%%%%%%%%%%%%%%%%%%%%%%%%%%%%%%%%%%%%%%%%%%%%%%%%%%%%%%%%%%%%%%%%%%%%%%%%%%%%%%%%%%%%%%%%%%%%%%%%%%%%%%%%%%%%%%%%%%%%%%%%%%%%%%%%%%%%%%%%
%%%%%%%%%%%%%%%%%%%%%%%%%%%%%%%%%%%%%%%%%%%%%%%%%%%%%%%%%%%%%%%%%%%%%%%%%%%%%%%%%%%%%%%%%%%%%%%%%%%%%%%%%%%%%%%%%%%%%%%%%%%%%%%%%%%%%%%%%%%%%%%%%%%%%%%%%%%%%%%%%%%%%%%%%%%%%%%%%%%%%%%%%%%%%%%%%%%%%%%%%%%%%%%%%%%%%%%%%%%%%%%%%%%%%%%%%%%%%%%%%%%%%%%%%%%%%%%%%%%%%%%

\begin{document}

\titlehead{\centering\normalfont\large\scshape Allgemeinverständliche Zusammenfassung}
\title{\Large\vspace{-\baselineskip} Wasser auf $\upalpha$-Aluminumoxid Oberflächen:\\
  Energetik, Dynamik and Kinetik\vspace{0\baselineskip}}
\author{\large\sffamily eingereicht von\\
  \Large\textbf\sffamily\ Sophia L. Heiden\\
  \large\sffamily (Universit\"{a}t Potsdam, Institut f\"{u}r Chemie)}
\date{}
\selectlanguage{german}
\maketitle
\vspace{-1cm}
Das wissenschaftliche Interesse an der Untersuchung von Oberflächen hat in den letzten Jahren stark zugenommen.
Oberflächen spielen unter anderem in Katalyse, Nanotechnologie und Korrosionsforschung eine wichtige Rolle.
Es wurden nicht nur Fortschritte im experimentellen Bereich, sondern auch in der theoretischen, computergestützten Analyse dieser Systeme erzielt.
Durch leistungsstärkere Computer und ausgefeiltere Software mit immer besseren Methoden können heutzutage wesentlich größere, komplexere Systeme mit höherer Genauigkeit untersucht werden, als noch vor zehn Jahren.

In dieser Arbeit wurden derartige Rechnungen angewandt, um Prozesse der $\upalpha$-Aluminiumoxid-Oberfläche besser zu verstehen.
Es wurde in drei Teilprojekten wissenschaftlichen Fragestellungen zu Aufbau, Stabilität, Wasseradsorption, Reaktivität und Schwingungseigenschaften nachgegangen, letztere auch im Vergleich zu experimentellen Befunden.

Das erste Teilprojekt befasste sich mit der (11\=20)-Oberfläche, zu der es bislang wenige Untersuchungen gibt.
Hier wurde zunächst die Oberfläche ohne Wasser %mithilfe periodischer Dichte\-funk\-tional\-theorie-Rechnungen
untersucht, um die Beschaffenheit zu erkunden.
Anschließend wurde das Verhalten eines Wassermoleküls auf der Oberfläche untersucht.
Es kann sowohl molekular adsorbieren, als auch in seine Bestandteile OH und H dissoziiert vorliegen, wobei die dissoziierten Strukturen wesentlich stabiler sind.
Die Reaktionsraten für Dissoziation und Diffusion wurden untersucht. Erstere sind sehr schnell (Größenordnung $10^{12}$ pro Sekunde) und letztere können einen weiten Bereich abdecken ($10^{-13}$-$10^6$s$^{-1}$).
Im Vergleich mit oberflächenspezifischen Schwin\-gungs\-spek\-tros\-ko\-pie-Experimenten konnte gute Übereinstimmung gefunden werden.
So waren wir in der Lage, die jeweiligen OD Gruppen jeder Schwingung den experimentellen Daten zuzuweisen, wobei D hier Deuterium, also schwerer Wasserstoff ist.

In einem zweiten Teilprojekt wurde auf ein Problem der genutzten Methodik eingegangen.
Wie aus der Literatur bekannt ist, werden bei dem Dichtefunktional PBE, das in dieser Arbeit hauptsächlich verwendet wurde, die Reaktionsbarrieren unterschätzt, und somit Raten überschätzt.
Mit Hilfe zweier unterschiedlicher Methoden konnte dieses Problem deutlich verbessert werden, sodass die Barrieren erhöht und die Raten verringert wurden, was mehr dem Bild der Realität entspricht.
Diese Methoden sind zum einen die sogenannten Hybridfunktionale und zum anderen lokale M\o{}ller-Plesset Störungstheorie.

In einem weiteren Projekt wurde die Streuung von Wasser an der Oberfläche untersucht.
In einem Molekularstrahlexperiment konnte kürzlich nachgewiesen werden, dass sich die Dissoziationswahrscheinlichkeit im Vergleich zur Pinhole-Dosierung erhöht (beides sind Methoden um Wasser auf die Oberfläche aufzubringen).
In dieser Arbeit konnte dies durch Simulationen nachgewiesen und mechanistisch aufgeklärt werden.

\end{document}
